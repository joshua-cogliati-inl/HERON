\section{ Installation and how to run}

\subsection{Installation}
As a plugin of RAVEN, HERON is installed as a submodule. RAVEN maintains up-to-date instructions for
plugin installation in its documentation
(\href{https://github.com/idaholab/raven/wiki/Plugins}{see the link to the Raven plugin
installation}.

\subsection{How to run}
Directly running HERON through RAVEN has not finished implementation. To run HERON, use bash to
run the executable:
\begin{lstlisting}
  /path/to/HERON/heron my_heron_input.xml
\end{lstlisting}
replacing
\begin{itemize}
  \item \texttt{/path/to/HERON} with your path to the HERON repository; and
  \item \texttt{my\_heron\_input.xml} with your HERON XML input.
\end{itemize}
For example, if my HERON is located as a submodule of RAVEN in \texttt{~/projects/raven}, and I
wanted to run the integration test \texttt{production\_flex}, I could do the following:
\begin{lstlisting}[basicstyle=\footnotesize]
  cd ~/projects/raven/plugins/HERON/tests/integration_tests/production_flex
  ~/projects/raven/plugins/HERON/heron heron_input.xml
\end{lstlisting}
Alternatively, you can use Python to run \texttt{HERON/src/main.py} with the HERON XML input as
argument; however, this will bypass loading the \texttt{raven\_libraries} and other initialization.

\subsection{Parallel Notes}

HERON uses RAVEN's parallel tools. Since running on different clusters
can require somewhat different commands, HERON allows the commands
used for parallel running to be chosen based on the hostname.

These are stored in the directory \texttt{templates/parallel}. Example:

\begin{lstlisting}[style=XML]
<parallel hostregexp="sawtooth[12].*">
  <useParallel>
    <mode>
      mpi
      <runQSUB />
    </mode>
  </useParallel>
  <outer>
    <parallelMethod>ray</parallelMethod>
  </outer>
</parallel>
\end{lstlisting}

The \texttt{hostregexp} is a regular expression and the first regular
expression that matches the hostname will be used as the template for
running in parallel.  If parallel is used, then the section in
\texttt{useParallel} will be added to the RunInfo in the RAVEN
input. If the batch size is greater than one then the code in the
section \texttt{outer} will be used.
